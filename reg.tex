\begin{frame}[allowframebreaks]{Reg Lez}

\begin{itemize}
  
\item lezione: Aspetti principali dell'esperimento ''paradigmatico'' di Rutherford al complesso degli acceleratori del Cern. Proiezione e commento di alcune slides: dall'esperimento di Rutherford (1910) al complesso degli acceleratori del Cern (2013). I costituenti elementari della materia ed i bosoni mediatori delle interazioni fondamentali nel Modello Standard. Dimensioni tipiche dall'atomo ai costituenti sub-nucleari. Cenno alla cronologia delle scoperte delle particelle elementari: elettrone, protone, muone, mesoni,....(le slides si trovano sul sito del corso in E-learning). 

\item lezione: Ricapitolazione della prima lezione a beneficio di alcuni studenti assenti. Cenno all'evoluzione dell' Universo nel modello del Big Bang. Condizioni di energia e tempo attualmente raggiunte all' LHC. Cenno alla ripartizione dell' energia dell' Universo tra Materia visibile ($5\%$ circa), Materia oscura ($25\%$), Energia Oscura ($75\%$). Necessit\'a di aumentare l'energia degli acceleratori per esplorare scale di distanze progressivamente pi\'u piccole. Indeterminazione di massa delle particelle instabili. Indeterminazione della posizione di una particella descritta da un'onda piana stazionaria.

\item lezione: La scoperta dell'antimateria. Dall'eqz. di Schroedinger (1925) per l'elettrone non relativistico all' eqz. di Dirac (1928) per l'elettrone relativistico (Griffiths). Cenno agli spinori ed alle matrici gamma nella rappresentazione di Bjorken e Drell. Cenno al fattore giromagnetico dell'elettrone ed alla difficoltà delle soluzioni con energia negativa. Scoperta del positrone di C. Anderson (1932) in camera a nebbia (Bettini). [Ref. sul sito del corso]

\item lezione: Identit\'a della rappresentazione mediante f.d.o. di una particella libera di energia E e della sua antiparticella con $E<0$. Rappresentazione grafica di Feynman dell'annichilazione di elettrone e positrone. Stato iniziale, finale ed intermedio: particelle reali e particelle virtuali. Non conservazione dell'energia negli stati intermedi per un intervallo di tempo dato dal principio di indeterminazione. Grafici di Feynman del decadimento beta del neutrone e del decadimento del muone 'spiegati' da un W virtuale di massa ca 82 GeV. Le f.d.o. di sistemi di particelle indistinguibili sono simmetriche o antisimmetriche sotto scambio delle particelle: bosoni e fermioni. Cenno alla Supersimmetria ed all'esistenza di partner supersimmetriche delle particelle reali.

\item lezione: Intensit\'a relativa delle 4 interazioni fondamentali: Forte, E.M., debole, Gravitazionale. Valore delle costanti di accoppiamento adimensionali. Valutazione della scala di energia a cui si dovrebbe manifestare il comportamento quantistico della gravit\'a: massa e lunghezza d'onda di Planck in analogia alla lunghezza Compton dell'elettrone. Cenno alla dipendenza dall'energia delle costanti di accoppiamento. Potenziale 'schermato' ipotizzato da Yukawa (1934), 'range' delle Int. Forti e massa attesa del mediatore. La scoperta del muone (1936) nei R.C. Cenno all'esperimento di CPP (1945): il muone non \'e il mediatore delle I.F. Scoperta del pione carico Powell et al. (1947) nei RC con la tecnica delle emulsioni fotografiche. Decadimento del pione carico in muone+neutrino.

\item lezione: L' esperimento di Conversi, Pancini, Piccioni (1946): il muone non \'e il mediatore delle I.F. Interazione Coulombiana, raggio di Bohr dell'atomo mesico, probabilit\'a di cattura del mesone negativo dai nuclei di Tomonaga,Araki (1940) in $10^{-12}s$, dipendenza da $Z^4$. Descrizione dell'apparato di misura: lenti magnetiche di Puccianti, coincidenze ritardate dei contatori di G.M., assorbitori con Z variabile. La scoperta del pione carico in emulsioni (Powell, 1947) e del pione neutro in camera a bolle (Steinberger et al. 1951). Slides e ref sul sito del corso.

\item esercitazione: Concetto di particella elementare che vive nello spazio minkowskiano. Ripasso delle trasformazioni di Lorenz. Il quadrivettore energia impulso. Massa, vita media e spin di una particella. Unit\'a di misura per massa ed energia. Calcolo della massa del protone in $MeV/c^2$. Connessione tra larghezza di riga e vita media di una particella. Calcolo della ''vita media'' della $\Delta^{++}$ a partire dalla sua larghezza. Cenni ad un quadrivettore per lo spin e sua connessione con l'elicit\'a. (Ref. Bettini Cap. 1) 

\item lezione: Definizione dell'operatore di Parit\'a spaziale e sua relazione con la riflessione speculare seguita da rotazione. Esempi di fdo (funzioni d'onda) pari, dispari, a parità non definita.  Parit\'a delle armoniche sferiche. Richiami alla parit\'a nelle transizioni di dipolo elettrico. Simmetria di  parit\'a per scambio nel caso di fdo fattorizzabili nelle parti spaziali, di spin, di isospin. Cenno alla generalizzazione del principio di Pauli per particelle ''indistinguibili'' secondo una data interazione. Esempio: il mesone vettore $\rho_0(770 MeV)$ non pu\'o decadere in due pioni neutri. Conservazione della parit\'a ed indistinguibilit\'a destra-sinistra nei sistemi fisici. Richiamo su quantit\'a scalari, pseudo-scalari, vettoriali e pseudo-vettoriali.

\item lezione: Spin Isotopico (Heisenberg,1932). Relazione tra carica elettrica, terza componente dell'Isospin $I_3$ e num. barionico. Esempi: Iso-singoletto $\lambda_0$, Iso-doppietto dei nucleoni, Iso-tripletto dei pioni, Iso-quadrupletto Delta, Nuclei C,N,O. Il Deutone D: principio generalizzato di Pauli e necessità del num. quantico di Isospin. Verifiche sperimentali: 1) rapporto delle sez. d'urto dei processi a) $p,p->D,\pi+$ b) $p,n->D,\pi0$ 2) rapporti delle sez. d'urto dei 6 processi $\pi,N$. Determinazione della parit\'a intrinseca $P_i$ del $\pi-$ (Chinowsky, Steinberger 1954) nel processo $\pi-,D->n,n$. RI-definizione della parit\'a $P=[(-1)^L]*P_i$. [Ref. su Isospin e su Parità intr. nel sito del corso]

\item lezione: Inclusione dell'Isospin nella simmetria per scambio. Cenno alla determinazione dello spin del pione carico e neutro. La scoperta dei mesoni K nei R.C.(Rochester e Butler,1946). Il $teta-Tau$ ''puzzle''. Ipotesi sulla violazione della parit\'a nelle Interazioni deboli di Lee,Yang 1956. [Ref. su violazione della Parità sul sito del corso]

\item esercitazione: Esempio di determinazione della parit\'a dello stato finale a 3 pioni carichi nel decadimento della $tau+$ nello stato $J^P=0^-$ . Introduzione all' esperimento di Wu et al. (1957): come individuare una direzione di riferimento rispetto alla quale misurare un'asimmetria destra-sinistra. Interazione di dipolo in campo esterno costante.

\item lezione: L'esperimento di M.me Wu (1957) sulla misura della violazione della parit\'a nelle interazioni deboli. Schema del decadimento beta con transizione di Gamow-Teller del $Co_{60}$ in $Ni_{60}$ senza cambiamento di parit\'a e con emissione di 2 fotoni. Definizioni dell' ANISOTROPIA dei fotoni e dell' ASIMMETRIA degli elettroni emessi. Descrizione dell'apparato sperimentale e del metodo di rivelazione di elettroni e fotoni. "Gedanken Experiment": l'asimmetria degli elettroni emessi implica anche la violazione della simmetria di coniugazione di carica C. Il prodotto delle simmetrie P e C ripristina la simmetria del sistema fisico nonostante ciascuna simmetria sia individualmente violata.

\item esercitazione: Genesi e significato dei diagrammi di Feynman. Generalizzazione relativistica della forza di Lorentz. Generalizzazione relativistica dell'equazione del moto di uno spin in campo elettromagnetico. L'equazione BMT. La precessione anomala. Definizione dell'anomalia del momento magnetico (g–2). Connessione con la misura dell'elicità di una particella. Cenni alla misura del (g–2). (ref. Jackson, 2nd ed, cap 11.11) Momento magnetico e momento angolare. Fattore giromagnetico. $G=2$ per fermioni elementari. G del protone e del neutrone. Ipotesi di Yukawa e mesone pesante mediatore delle interazioni forti. Cenni alla derivazione del potenziale di Yukawa. Spiegazione di Wick del momento magnetico anomalo di protone e neutrone. Calcolo dei tempi caratteristici dell'interazione. (ref. K. Nishijima, ''Fundamental Particles'', cap. 1)

\item esercitazione: Cinematica dei decadimenti a 2, 3 e n-corpi. Monocromaticit\'a delle particelle figlie nei decadimenti a 2 corpi. Differenza tra energia totale ed energia cinetica (misurabile). Esempi notevoli di decadimenti a due corpi e calcolo dell'energia delle particelle nello stato finale. Energia dei fotoni nel decadimento del $\pi0$, energia di muone e positrone nel decadimento del pione carico. Esempio di transizione nucleare; il caso del Samario 152. Energia dei fotoni emessi. Cenno al rinculo dell'atomo. Decadimenti a tre corpi a partire dal decadimento del muone. Energia massima e minima del positrone. Stato dell'emissione dei neutrini nel caso di Emax e Emin. Decadimento a n-corpi. Energia massima e minima per una particella nel decadimento a n-corpi. Energia nel sistema di riferimento di laboratorio. Spettro dei fotoni da decadimento del pi0. Isotropia implica spettro piatto.

\item lezione: Definizione dell'operatore unitario di coniugazione di carica C sul corpo dei complessi. Fasi degli autovalori di C e di P. Gli autostati sono bosoni neutri: il caso del pione e del fotone. Valore di C per sistemi di fermione,antifermione o bosone-antibosone di mom. ang. L e di spin tot. S (Bettini). Definizione dell'operatore di G parità per un ISO-multipletto di n particelle. Nomenclatura $I^G$, $J^{(PC)}$. Applicazione ai decadimenti in 2 o 3 pioni delle particelle $1^{(--)}$ $\rho(770)$, $\omega(782)$, $\phi(1020)$, $J/\Psi(3100)$.

\item esercitazione: Memorizzazione delle propriet\'a principali delle particelle elementari fondamentali. Leptoni e quark, mesoni e barioni. Simmetrie e leggi di conservazione. Esempi di decadimenti proibiti e permessi in base a conservazione di energia-impulso, momento angolare e principali numeri quantici (Q, B, L, Lf, S).

\item lezione: L'esperimento di M.me Wu (1957) sulla misura della violazione della parit\'a nelle interazioni deboli. Schema del decadimento beta con transizione di Gamow-Teller del $Co_{60}(5+)$ in $Ni_{60}(4+,2+,0+)$ senza cambiamento di parit\'a e con emissione di 2 fotoni. Definizioni dell'ANISOTROPIA dei fotoni e dell'ASIMMETRIA degli elettroni emessi. Descrizione dell'apparato sperimentale, del metodo di individuazione di una direzione definita e del metodo di rivelazione di elettroni e fotoni. "Gedanken Experiment": l'asimmetria degli elettroni emessi implica anche la violazione della simmetria di coniugazione di carica C. Il prodotto delle simmetrie P e C ripristina la simmetria del sistema fisico nonostante ciascuna delle 2 simmetrie sia individualmente violata. Articolo, schema di decadimento del $C0_{60}$ sul sito del corso

\item lezione: Produzione "associata" di particelle "strane". Il processo $\pi^- P ->K_0 + \Lambda_0$ genera mediante interazione forte particelle che decadono debole con vita media $\tau\approx 10^{-10} s$. Definizione di Iperone e propriet\'a della $\Lambda_0$. Nuovo numero quantico di stranezza e definizione di Ipercarica (Gellman, Nishijima 1953). Due doppietti di Isospin per i mesoni K neutri e carichi. Cenno alla composizione in termini di quarks. Peculiarit\'a della ''vita media'' $\tau$ del mesone $K_0$ di cui vengono misurati due valori: $\tau_{short}=O[10^{-10}s]$ e $\tau_{long}=O[10^{-8}s]$ corrispondenti rispettivamente a decadimenti in 2 corpi ed in 3 corpi. 

\item lezione: I mesoni K hanno parit\'a intrinseca negativa. Effetto di CP sui mesoni $K0$ e $K0_{bar}$. Costruzione degli Auto Stati di CP $K1$ e $K2$ con A.V. $\pm1$. B.R. di $KS$ in 2 corpi e di $KL$ in 3 corpi tra cui decadimenti semi-leptonici. L'esperimento sulla misura della violazione di CP (Cronin et al. 1964) Distribuzione dell' articolo: descrizione dell'apparato sperimentale e del metodo di misura. Cenni al funzionamento delle ''spark chambers'' e dei Cerenkov a soglia ad $H2O$. Selezione dei decadimenti a 2 ed a 3 corpi basata sull'angolo rispetto alla linea di volo del $K0$. Determinazione del fattore gamma del $K0$ e valutazione del contributo del $KS$ a 19 m. dal bersaglio.

\item lezione: Regola empirica $\Delta S=\Delta Q$ nei decadimenti SL dei K. Asimmetria di carica nel decadimento S.L. del KL. Cenno al meccanismo della rigenerazione. Misura dell'interferenza tra KS rigenerati e decadimenti $KL\to\pi+\pi-$ \'e lo stesso stato fisico. Cenno al formalismo dell'evoluzione temporale di KS e KL, A.S. di massa e di vita media. L'Hamiltoniana di evoluzione temporale \'e la matrice $2x2$ $(M-i\Gamma)$. La violazione di CP misurata nell'esperimento di Cronin interviene nell'Hamiltoniana di evoluzione temporale (detta nel mescolamento) e non in quella di decadimento (detta diretta). Cenno alle misure della violazione nel mixing e diretta nel settore dei mesoni neutri $K$ e $B$. 

\item esercitazione: Calcolo della parit\'a di un generico sistema di 2 particelle nel CM. La parit\'a di due pioni carichi \'e $0+,1-,2+$ e se provenienti dal decadimento di un $K0$ \'e $0+$. Utilizzando il principio di Pauli generalizzato si ottiene lo stesso risultato. Estensione banale al caso di 2 pioni neutri. Il valore massimo dell' Isospin di un mesone \'e 1 e quello di un barione \'e $3/2$. Esempi: doppietti dei mesoni K, tripletto dei pioni, quadrupletto dei barioni $\Delta(1232)$. Cenno alla composizione in termini di quark.

\item lezione: Simmetrie. Definizione operativa di simmetria per un dato sistema fisico. Simmetrie statiche (es. le rotazioni), dinamiche associate al moto di un sistema (es. invarianza temporale T) Simmetrie continue e discrete (es. C,P,T). Le rotazioni costituiscono un ''gruppo'' abeliano in 2D, non abeliano in 3D. Cenno alla rappresentazione matriciale; cenno alle simmetrie del Modello Standard $U(1)\times SU(2)\times SU(3)$. Cenno a simmetrie, invarianza della Lagrangiana e leggi di conservazione: enunciato del Teorema di N\:others (1917) per simmetrie continue. Quantit\'a fisiche conservate e quantit\'a non-osservabili (Ref. Griffiths, Perkins).

\item esercitazione: Disegno del diagramma \'a la Feynman del processo di produzione e decadimento della $Y*$. Estrazione del rapporto fra le sezioni d'urto di produzione di $Y*$ negativa e positiva a partire dal grafico. Calcolo delle energie massime e minime dei pioni prodotti nel suo decadimento. Vite medie forti ed e.m. ad uno o pi\'u fotoni (es. $\Sigma_0\to\Lambda_0 \gamma$, $\pi0->\gamma \gamma$). Decadimenti possibili delle $\Sigma(1385)$ e $\Sigma(1185)$.

\item esercitazione: Determinazione della parit\'a di un sistema a 3 pioni. Caso generale e caso di 3 pioni complessivamente neutri da decadimento di $K_L$. Determinazione della CP parit\'a di 2 pioni e di 3 pioni provenienti da decadimenti rispettivamente del $K_S$ e del $K_L$. Lo spazio delle fasi disponibile per decadimenti del $K_L$ in 3 pioni tutti in onda P \'e molto ridotto (Ref. Bettini). L'eventuale CP parit\'a positiva dello stato finale $\pi+\pi-\pi0$ non influenza le conclusioni dell'esperimento di Cronin et al. sulla violazione di CP.

\item  lezione: Enunciato del teorema di CPT (Schwinger, L\:uders, Pauli) e delle 5 ipotesi su cui \'e basato. Conseguenze sperimentalmente verificabili su massa, vita media, carica, mdm di particelle ed antiparticelle. Limiti superiori su diff. di massa $M(K_0)-M(K_0bar)$, $mdm(elettrone)-mdm(positrone),$ rapporto $Q/M$ del protone vs anti-protone. I momenti di dipolo elettrico e magnetico di una particella con spin devono essere allineati allo spin. Il mde di una particella con spin v\'iola le simmetrie di parit\'a P e di inversione temporale T (Griffiths).

\item  lezione: Fenomenologia dei leptoni carichi: elettrone, muone, tau. Ipotesi di Pauli (1930) sull'esistenza del 'neutrino' per spiegare come lo spin dei nuclei di $\cel{Li}{6}{}{}$ ed $\cel{N}{14}{}{}$ sia intero e non semi-intero come previsto dal modello nucleare dell'epoca (Bettini). Sezione d'urto del processo beta inverso ($\nu p\to e+n$) su protone libero di Bethe-Peierls (1934). Cenno al processo di fissione indotta da neutrone su $\cel{U}{235}{}{}$ con emissione in media di 6 anti-nu e valutazione del flusso di anti-nu da reattore nucleare da 3 GW. Velocità di reazione e stima della massa di H20 necessaria. Gli esperimenti di Reines\&Cowan al reattore di Savannah River (1953). Descrizione dell'apparato, del metodo di misura del positrone e del neutrone. Verifiche sperimentali. Cenno ai risultati negativi dell'esperimento di radio-chimica di Davies eseguito a Savannah River.

\item  Presentazione di slides sull'esperimento di Reines-Cowan. Visione e commento di un breve filmato dell'INFN sui neutrini (vedi sito del corso). Cenno all' esperimento di BNL-Columbia (1962) sulla determinazione dell'esistenza del neutrino muonico. L'esperimento di Davies $\nu_e+Cl_{37}\to Ar_{37}+e$ fornisce risultati positivi nella misura dei neutrini solari. I $\nu_e$ solari sono diversi dagli anti $\nu_e$ da reattori nucleari. Cenno alla distinzione tra neutrini di Dirac e neutrini di Majorana.

\item  lezione: Introduzione all'esperimento di Goldhaber (1957) sulla misura indiretta dell'elicit\'a del neutrino. Decadimento del $Eu_{152}(J=0^-)$ per cattura elettronica in $(Sm_{152})^*(J=1^-)+\nu_e(840KeV)$ e successivo decadimento in $Sm_{152}+\gamma(960KeV)$. L'elicit\'a del $\gamma$ \'e identica a quella del $\nu_e$. I fotoni emessi lungo la linea di volo del $Sm*$ presentano il max. valore del boost di energia.

\item  lezione: Transizione energetica del $Sm*$: calcolo dell' energia del fotone emesso ed energia di rinculo del nucleo di $Sm_{152}$. Vita media del $Sm*$, calcolo della larghezza di riga intrinseca ed allargamento di riga dovuto al Doppler termico. Calcolo del boost Doppler dei fotoni emessi lungo la linea di volo del $Sm*$ e prima che il $Sm*$ possa perdere energia. La sovrapposizione parziale degli spettri di emissione e di assorbimento permette l'assorbimento risonante. Descrizione dell'apparato e del metodo di misura di Goldhaber et al. Cenno alla dipendenza della sez. d'urto Compton su Fe magnetizzato dalla polarizzazione del fotone: spin-flip degli elettroni con spin antiparallelo alla polarizzazione del fotone.

\item esercitazione: Fenomeni di oscillazione in fisica delle particelle. Il $K0$ e il $K0$bar possono decadere nello stesso stato. CP parit\'a dello stato a due pioni. C e P di coppie di bosoni e fermioni identici. Definizione di $K1$ e $K2$. Diversa massa e vita media. Connessione della differenza di massa con gli elementi fuori diagonale dell'Hamiltoniana. Calcolo della formula dell'oscillazione di $K0$ in funzione del tempo nell'approssimazione di vita media infinita. Cenni alla rigenerazione dei K. Cambiamento di sapore nei neutrini. Oscillazioni nel vuoto fra le tre famiglie. Calcolo dell'oscillazione nell'approssimazione di mixing di due famiglie, esempio del caso mu-tau. (Rif. Okun Leptons and Quarks, cap. 10 e 11, Bettini cap. 8 e 10)

\item lezione: Il valore della costante di accoppiamento debole estratto dal decadimento del muone mediante la regola d'oro di Fermi \'e leggermente diverso da quello ottenuto dal decadimento beta del neutrone, ed \'e ca 20 volte quello ottenuto dal decadimento della $\sigma-\to n\pi-$. Ipotesi di Cabibbo (1963): gli A.S. delle W.I. sono una c.l. degli A.S. delle S.I.(Perkins). Ipotesi G.I.M. (1970) dell'esistenza di un quarto quark per spiegare, tra l'altro, l'assenza del decadimento $K0\to\mu+\mu-$. Cenno ai ''box diagram'' di cancellazione. Cenno all'ipotesi di Kobayashi-Maskawa (1972): l'esistenza di 3 famiglie di quark permette la spiegazione della CP violation. Parametri reali e fasi complesse indipendenti in una matrice unitaria $3x3$. Una fase ineliminabile viola T e CP. (Ref Griffiths Cap. 9).

\item  lezione: Produzione adronica agli anelli di accumulazione e+e- ad energie nel C.M. fino a 3 GeV: mesoni $\rho(770),\omega(787),\Phi(1020)$. Rapporto R tra $\sigma(e+e\to\text{adrons})/\sigma(e+e\to\mu+\mu-)$. Cenno alla Vector Meson Dominance. R(E) in funzione della carica frazionaria dei quark accessibili all'energia nel C.M. Cenno alla produzione di Jets a Petra ed alla loro distribuzione angolare $[1+\cos^2(\theta)]$. La rivoluzione della 'nuova fisica': la scoperta della $J/\Psi$ e del quarto quark charm (1974). L'esperimento di S. Ting a BNL e l'esperimento di B. Richter a SLAC. Formula di Breit-Wigner per il processo ($e+e\to J/\Psi\to e+e-$). Misura dell'altezza della $J/\Psi$ rispetto al 'continuo' e della sua larghezza $\Gamma$; valutazione della sua vita media $\tau$. Confronto con le $\Gamma$ di $\rho,\omega,\phi$. Cenno agli stati eccitati della $\Psi$ ed alla scoperta della Yupsilon(9460) e del quinto quark beauty. (Ref. Perkins + slides sul sito del corso).

\item esercitazione: Richiamo sui livelli energetici di un sistema sottoposto a potenziale coulombiano. Massa ridotta e raggio di Bohr. La funzione d'onda del positronio e i suoi livelli in relazione al quarkonio. Somiglianza dei livelli energetici del sistema $e+e-$, $c-cbarra$ e $b-bbarra$. Potenziale tra quark. Termine coulombiano piu' termine lineare. Valori di best fit di $alpha_s$ e kappa. Disegno del potenziale e sua relazione coi raggi del sistema q qbarra. Significato fisico e pittorico del termine lineare. Confinamento dei quark. Stima di $alpha_s$. La regola di Ozi e sua spiegazione semi-quantitativa.

\item esercitazione: Fattore di Land\'e g ; effetto Zeeman e effetto Zeeman anomalo. Cenni all'equazione di Dirac e al termine con $g=2$. Precessione di spin in campo magnetico e frequenza di Larmor $w_L$. Particella carica in campo magnetico (forza di Lorentz) nel caso relativistico e frequenza di ciclotrone $w_C$. Precessione di Thomas (trattazione semplificata). Moto relativo spin-momento nel caso relativistico: $w_A = w_L - w_T - w_C$. Proporzionalita' con $a_mu$ Polarizzazione del $mu+$ da decadimento del pi+. Asimmetria in direzione dello spin del positrone da decadimento di mu+. Muone mu in storage ring e "wiggle plot" (=decadimento+oscillazione con frequenza $w_A$). (Ref. Slides nel sito del corso)

\item esercitazione: Limiti alla massa del neutrino dal decadimento beta del Trizio. Schema di decadimento. Calcolo del Q-valore. Differenza tra masse atomiche e nucleari. Calcolo della probabilit\'a di decadimento a partire dalla regola d’oro di Fermi. Elemento di matrice. Cenno al calcolo dell’elemento di matrice. Calcolo dello spazio delle fasi nel caso $m(nu)=0$ e $m(nu)\neq0$. Caratteristiche dell’end-point. Effetto della risoluzione sperimentale sulla misura. (Ref. [1] E. Fermi, Nuclear Physics, University of Chicago Press. [2] KATRIN Letter of Intent) 

\end{itemize}

\end{frame}