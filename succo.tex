\section{Physical quantities:unit\'a di misura e ordini di grandezza e scaling laws.}

\subsection{masses}

\begin{frame}{elementar masses}
\begin{columns}[T]
\begin{column}{0.5\textwidth}
\begin{itemize}
\item $m_P=938.3\Mcs$
\item $m_N=939.7\Mcs$
\item $m_e=0.51\Mcs$
\item  $1 u=931.49 \Mcs$.
\end{itemize}
\end{column}
\begin{column}{0.5\textwidth}
\begin{itemize}
    \item $m_{\Ppi}\approx139p$
    \item $m_{\Ppi}\approx134$
\end{itemize}
\end{column}
\end{columns}
\end{frame}

\subsection{Interaction coupling}

\begin{frame}{costante struttura fine}
    $\alpha_{SF}=\frac{e^2}{(4\pi\epsilon_0)\hbar c}\approx\frac{1}{137}$.
\end{frame}

\subsection{Energie}

\begin{frame}{Energia in varie unit\'a}

Relazioni $u,c,MeV$:
 
$c^2=931.502\frac{MeV}{u}$.
 
$e^2=1.44\,MeV\,fm$.
 
Conversione eV-Kelvin:
 
\begin{align*}
1  ^{\degree}K&= 8.621738*10^{-5}  eV \\
&= 0.0862 meV \\
&= 0.695 cm^{-1}
\end{align*}

\begin{align*}
1 a.u=27.211396 eV=219474.63 cm^{-1}\\
1 Ry=13.6057 eV \\
1 eV =8065.54 cm^{-1} \\
1 eV= 11,600  ^{\degree}K\\
1 meV = 8.065 cm^{-1}
\end{align*}
\end{frame}

\subsection{lengths}

\begin{frame}{Borh radius, electron radius, Compton wavelength}
    The ratio of three characteristic lengths:

the classical electron radius $r_e=(\frac{1}{4\pi\epsilon_0})\frac{e^2}{m_ec^2}$, the Bohr radius $a_0=(4\pi\epsilon_0)\frac{\hbar^2}{m_ee^2}$  and the Compton wavelength of the electron $\lambdabar_e=\frac{\hbar}{m_ec}$:

$r_e=\frac{\alpha\lambda_e}{2\pi}=\alpha^2a_0$.
\end{frame}

\begin{frame}{Atom vs nucleus}
\begin{columns}[T]
\begin{column}{0.5\textwidth}
Atomo d'idrogeno:
    \begin{itemize}
        \item Raggio di Bohr.
$a_0=\frac{\hbar^2}{m_ee^2}=5.3*10^{-9} cm$
\item Magnetone di Bohr.
SI: $\mu_B=\frac{e\hbar}{2m_e}=5.788*10^{-5}eV*T^{-1}$.
cgs: $\mu_B=\frac{e\hbar}{2m_ec}$.
atomic unit: $\frac{\alpha}{2}$
\item  Livelli energetici $E_n=\frac{m_ec^2\alpha^2}{2n^2}=\frac{13.6eV}{n^2}$.
    \end{itemize}
\end{column}
\begin{column}{0.5\textwidth}
    \begin{itemize}
        \item    raggio nucleare
        \item Magnetone nucleare
SI $\mu_N=\frac{e\hbar}{2m_pc}=3.15245*10^{-8}eV*T^{-1}=\frac{\mu_B}{1836}$.
\item Momento di quadrupolo.
\item Deutone.
\begin{align*}
&B=2.22 MeV\\
&\mu_d=0.8574376\mu_N\\
&S=1\\
&Q=2.8*10^{-27}cm^2=0.28fm^2(Q_{zz})\\
&=\frac{2}{5}eZ(b^2-a^2)
\end{align*}
\item barn $=10^2 fm^2=10^{-24}cm^2$
    \end{itemize}
\end{column}
\end{columns}
\end{frame}